\documentclass[12pt]{article}
\usepackage[left=2cm,right=2cm,top=2cm,bottom=2cm]{geometry}
\usepackage{wrapfig}
\usepackage{booktabs}
\usepackage{graphicx}
\usepackage{amssymb}
\usepackage{siunitx} % Required for alignment
\usepackage{subfigure}
\usepackage{multirow}
\usepackage{rotating}
\usepackage[T2A]{fontenc}
\usepackage[russian]{babel}
\usepackage{caption}
\usepackage{hyperref}
\usepackage{mathtools}
\usepackage{amsmath}
\usepackage{float}

\hypersetup{
    colorlinks=true,
    linkcolor=blue,
    urlcolor=magenta,
    citecolor=blue,
    }
\graphicspath{{pictures/}}

\begin{document}
  \begin{titlepage}
      \begin{center}
          \vspace*{1cm}

          \Huge
          \textbf{ДИАМАГНЕТИЗМ И ПАРАМАГНЕТИЗМ}

          \vspace{1.5cm}

          \Large
          \textbf{Балдин Виктор Б01-303}

          \vfill

          Вопрос по выбору \\
          Устный экзамен по общей физике

          \vspace{0.8cm}

          \includegraphics[width=0.4\textwidth]{university_logo.png}

          Физтех-школа радиотехники и компьютерных технологий\\
          Московский физико-технический институт\\
          Долгопрудный, 2024
      \end{center}
  \end{titlepage}

  \tableofcontents

  \section{Введение}
  Переходя к объяснению магнитных свойств материальных сред с атомистической точки зрения, заметим прежде всего, что в последовательно классической теории магнетизм должен отсутствовать. Бор в 1911 г. и независимо от него Ван-Лёвен в 1920 г., пользуясь методами классической статистики, строго доказали следующую теорему. В состоянии термодинамического равновесия система электрически заряэненных частии (электронов, атомных ядер и пр.), помещенная в постоянное магнитное поле, не могла бы обладать магнитным моментом, если бы она строго подчиняласъ законам классической физики. Такая система может быть намагничена только в неравновесном состоянии. Если она перейдет в равновесное состояние, то намагничивание исчезнет. Причина этого, грубо говоря, заключается в том, что постоянное магнитное поле, действуя на заряженную частицу с силой, перпендикулярной к скорости, не может изменить кинетическую энергию частицы. Для объяснения магнетизма вещества требуется привлечение квантовых представлений.

  Между тем парамагнетизм и диамагнетизм были объяснены, и притом довольно успешно, Ланжевеном (1872-1946) в 1905 г. без использования квантовых представлений. Причина этого заключается в том, что в классических теориях намагничивания молчаливо вводились представления сугубо квантового характера. Именно, предполагалось, что из электрически заряженных частиц можно построить устойчивые образования - атомы и молекулы. От последовательно классической теории надо требовать объяснения не только намагничивания, но и существования самих атомов, что удалось сделать только квантовой механике. Поскольку последняя в нашем курсе еще не излагалась, при объяснении намагничивания мы будем пользоваться полуклассическими представленияли. Несмотря на свою непоследовательность и недостаточность, полуклассическая теория позволяет в основном уяснить природу намагничивания.

  Начнем с краткого рассмотрения магнитных свойств атомов. Более подробно этот вопрос будет разобран в т. V нашего курса - в атомной физике. В простейшей боровской модели атома водорода электрон вращается вокруг ядра по окружности. Заряд злектрона будем обозначать через $-\epsilon$. Вращающийся по окружности электрон в среднем возбуждает магнитное поле как ток $I=-e / T$, где $T=2 \pi r / v-$ период обращения электрона. Поэтому вращающемуся электрону присущ не только орбитальный момент импульса (или механический момент) $L=$ $=m r v$, но и магнитный момент $\mathfrak{M}=I S / c=-e r v /(2 c)$. Отношение этих величин называется гиромагнитньм отношением и для нашей модели атома равно

  $$
  \Gamma=\frac{\mathfrak{N}}{L}=-\frac{e}{2 m c}
  $$


  Тот же результат справедлив для движений злектрона по эллиптическим орбитам. Он верен и для многоэлектронных атомов, поскольку для всех электронов отношение $e / m$ одно и то же.

  Согласно теории Бора момент импульса атома кбантуется, т.е. может принимать не непрерывный, а только дискретный ряд значений. Допустимыми являются значения $L=n \hbar$, где $n$ - целое число, которое может принимать значения $1,2,3, \ldots$, а $h=h /(2 \pi)=$ $=1,05 \cdot 10^{-27}$ эрг $+\mathrm{c}-$ постоянная Планка (1858-1947), деленная на $2 \pi$. (Эта величина также называется постоянной Планка и более удобна в теоретических вопросах.) Вместе с механическим моментом магнитный момент также квантуется в соответствии с формулой

  $$
  \mathfrak{m}=-\frac{e \hbar}{2 m c} n
  $$


  Таким образом, наименышее значение магнитного момента атома равно

  $$
  \mathfrak{M}_{\mathrm{B}}=\frac{e \hbar}{2 m c}=9,28 \cdot 10^{-21} э \mathrm{pr} / \mathrm{Fc}
  $$

  Эта величина играет роль атома магнитного момента и называется магнетоном Бора.
  Квантовая механика оставила представление о движении электронов по классическим орбитам и уточнила правила квантования теории Бора. Вместо движения самих электронов квантовая механика ввела представление о движении некоторой величины, имеющей смысл плотности вероятности нахождения электрона в пространстве. Классическим, однако отнюдь не адекватным аналогом такого представления, может служить облако, в котором масса и соответствующий ей элктрический заряд распределены в пространстве непрерывно с определенной плотностью. Существует дискретный ряд так называемых стауиолариых состояний, в которых эти величины не меняются во
  \section{Диамагнетизм}

  \section{Парамагнетизм}

  \section{Заключение}

% \bibliography{references}
% \bibliographystyle{gost}

\end{document}
