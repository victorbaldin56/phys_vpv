\documentclass[12pt]{article}
\usepackage[left=2cm,right=2cm,top=2cm,bottom=2cm]{geometry}
\usepackage{wrapfig}
\usepackage{booktabs}
\usepackage{graphicx}
\usepackage{amsmath}
\usepackage{siunitx} % Required for alignment
\usepackage{subfigure}
\usepackage{multirow}
\usepackage{rotating}
\usepackage[T2A]{fontenc}
\usepackage[english, russian]{babel}
\usepackage{caption}
\usepackage{hyperref}
\usepackage{float}

\hypersetup{
    colorlinks=true,
    linkcolor=red,
    urlcolor=magenta,
    }
\graphicspath{{pictures/}}

\begin{document}
    \begin{titlepage}
        \begin{center}
            \vspace*{1cm}

            \Huge
            \textbf{ИЗМЕРЕНИЕ ТЕПЛОПРОВОДНОСТИ ВОЗДУХА ПРИ РАЗНЫХ ДАВЛЕНИЯХ}

            \vspace{1.5cm}

            \Large
            \textbf{Комкин Михаил Б01-303}

            \vfill

            Вопрос по выбору \\
            Устный экзамен по общей физике

            \vspace{0.8cm}

            \includegraphics[width=0.4\textwidth]{university_logo.png}

            Физтех-школа радиотехники и компьютерных технологий\\
            Московский физико-технический институт\\
            Долгопрудный, 2024
        \end{center}
    \end{titlepage}
    
    \begin{itemize}
        \item{Цель работы:} исследовать теплопередачу от нагретой нити к цилиндрической оболочке в зависимости от концентрации\\ 
        (давления) заполняющего её воздуха. Измерить коэффициент теплопроводности при высоких давлениях; определить область перехода к 
        режиму теплопередачи; определить коэффициент теплопередачи при низких давлениях.\\
        \item{В работе используются:} цилиндрическая колба с натянутой по оси платиновой нитью; форвакуумный насос; вакуумметр; масляный манометр; вольтметр и амперметр
        (цифровые мультиметры); источник постоянного тока.
    \end{itemize}
    \section{Теоретические сведения}  
    
    \section{Установка}
    
    \section{Ход работы}     
        \begin{enumerate}
            \item Проведем предварительные расчеты параметров опыта. Приняв газокинетический диаметр молекул равным $d \sim 3.5 A$, оценим
            длину свободного пробега молекул при атмосферном давлении.\\
            Оценим, при каком давлении $P_1$ длина свободного пробега сравняется с радиусом нити. В единицах масляного столба: $P_1 \approx 500 \text{мм.масл.ст.}$
            \item  Зафиксируем данные установки:
            \[2r_\text{н} = 0,05~\text{мм}\quad 2R = 10~\text{мм} \quad L = (222\pm 2)~\text{мм}\]
            \item Запишем значения атмосферного давления $P_{\text{атм}}$ и температуры $t_{\text{к}}$ в комнате.
            \item Убедимся, что перед началом эксперимента установка находится под вакуумом. Кран $K_1$ открыт, $K_2$ — закрыт, $K_3$— открыт.
            \item Запустим воздух в установку, плавно открывая кран $K_2$, включим в сеть цифровые мультиметры. Установите амперметр в режим
            измерения постоянного тока, а вольтметр — постоянного напряжения.
            \item 
            
                    
        \end{enumerate}

\newpage
\section{Приложение 1}

\end{document}
